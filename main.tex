\documentclass{article}
\usepackage[utf8]{inputenc}

\title{bioinf}
\author{Ivan Mihajlin}
\date{July 2018}

\usepackage{natbib}
\usepackage{graphicx}

\begin{document}

\maketitle

\section{Formulation closest to the actual process}

Bellow, we provide the description of a valid phylogenetic tree with references $S_1 \dots S_k \in A^p$, where $A$ is the alphabet:

A phylogenetic tree with references $S_1 \dots S_k \in A^m$ is a tree where each node has a label $\in A^q$. For each node except for the root, the label can be produced from its par rent's label by replacing a substring with a substring (of the same size?) from one of the reference strings.

The phylogenetic tree with references recovery problem can be described as following: one is given a set $R$ of reference strings as well as an additional set $T$ of strings. One needs to output the best under some parameterization phylogenetic tree with references $S_1 \dots S_k$, that has every string from $T$ as a label of some of the nodes. 


\section{More sloppy formulation}

In this section, we are trying to provide a less consistent with the actual biology, but more tractable algorithmically formulation. We will assume that references are distinct enough so we can always understand which piece of a label is coming from which reference:

A phylogenetic tree with implicit references is a tree where each node has a label $\in [0 \dots k]^*$. The root is labeled with $"0"$. For each node except for the root, the label can be produced from its parent's label by replacing a substring (including a zero size substring) with a symbol from $[1 \dots k]$.

The phylogenetic tree with implicit references recovery problem can be described as following: one is given a set $T$ of strings $\in [0 \dots k]^*$. One needs to output the best under some parameterization phylogenetic tree with implicit references, that has every string from $T$ as a label of some of the nodes. 


\end{document}

